\begin{enunciado}{\ejercicio}

  Dadas las funciones booleanas \texttt{F} y \texttt{G}
  definidas a partir de las siguientes tablas de verdad:\par

  \begin{center}
    {        \ttfamily
      \begin{tabular}{|c|c|c|c|}
        \hline
        A & B & C & F(A,B,C) \\ \hline
        1 & 1 & 0 & 0        \\ \hline
        1 & 0 & 0 & 1        \\ \hline
        1 & 0 & 1 & 1        \\ \hline
        0 & 1 & 0 & 0        \\ \hline
        0 & 0 & 1 & 0        \\ \hline
        0 & 1 & 1 & 1        \\ \hline
        1 & 1 & 1 & 1        \\ \hline
        0 & 0 & 0 & 0        \\ \hline
      \end{tabular}
    }
    {        \ttfamily
      \begin{tabular}{|c|c|c|c|}
        \hline
        A & B & C & G(A,B,C) \\ \hline
        0 & 0 & 0 & 1        \\ \hline
        0 & 1 & 0 & 0        \\ \hline
        0 & 0 & 1 & 1        \\ \hline
        0 & 1 & 1 & 1        \\ \hline
        1 & 0 & 0 & 1        \\ \hline
        1 & 1 & 0 & 1        \\ \hline
        1 & 0 & 1 & 0        \\ \hline
        1 & 1 & 1 & 1        \\ \hline
      \end{tabular}
    }
  \end{center}

  \begin{enumerate}[label=\alph*)]
    \item Escribir la \texttt{suma de productos} para ambas funciones. Calcular la cantidad
          de compuertas que la implementación literal requeriría en cada caso.

    \item ¿Se pueden simplificar las expresiones usando propiedades del álgebra booleana?
          Para cada función decidir si es posible y, en caso de que lo sea, dibujar el circuito utilizando
          la menor cantidad de compuertas que pueda.
  \end{enumerate}

\end{enunciado}

\begin{enumerate}[label=\alph*)]
  \item Eligiendo las líneas que tienen 1s en la función:
        $$
          \texttt{F = A\negado{B}\negado{C} + A\negado{B}C + \negado{A}BC + ABC }
        $$
        Con esta descripción necesitaría cuatro compuertas AND de 3 entradas y 1 OR de 4 entradas.

  \item ¿Se pueden simplificar las expresiones usando propiedades del álgebra booleana?
        Para cada función decidir si es posible y, en caso de que lo sea, dibujar el circuito utilizando
        la menor cantidad de compuertas que pueda.
\end{enumerate}

