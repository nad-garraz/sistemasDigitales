Todas las compuertas mencionadas en esta práctica son de 1 o 2 entradas, a menos que se indique lo contrario.
Usaremos símbolos detallados a continuación para representar las distintas funciones lógicas:
{\tt XOR $\to \oplus$, NAND $\to |$, NOR $\to \downarrow$   }.\par
Durante la presente práctica se recomienda fuertemente la utilización de un simulador para experimentar con
los componentes y cicuitos propuestos y verificar las soluciones. Una recomendación es el \href{http://www.cburch.com/logisim}{\texttt{Logisim} (http://www.cburch.com/logisim/)}\par\bigskip

\section*{\tt Circuitos Combinatorios}

\begin{enunciado}{\ejercicio}
  Demostrar si las siguientes equivalencias de fórmulas booleanas son verdaderas o falsas:
  \begin{enumerate}[label=\tt\alph*)]
    \item {\tt x\por z=(x+\negado{\texttt{y}})\por (\negado{x}+z)}.
    \item {\tt x $\oplus$ (y\por z) = (x $\oplus$ y)\por (x $\oplus$ z)} donde se aplica la propiedad distributiva con respecto a $\oplus$.
  \end{enumerate}
\end{enunciado}

\begin{enumerate}[label=\tt\alph*)]
  \item
        {\tt
        (x+y)(x+\negado{y})(\negado{x}+z) =
        (xx+x\negado{y}+xy+y\negado{y})(\negado{x}+z) =
        (x+x(y+\negado{y})+0)(\negado{x}+z) =
        x(\negado{x}+z) =
        x\negado{x}+xz
        $\igual{\checkmark}$
        xz
        }
  \item {\tt x$\oplus$(yz) = (x$\oplus$y)(x$\oplus$z)} donde se aplica la propiedad distributiva con respecto a $\oplus$.\par
        \begin{center}
          {\tt
            \begin{tabular}{|c|c|c|c |c|c|c|c}
              \cline{1-8}
              x & y & z & yz & x $\oplus$ (yz) & x$\oplus$y & x$\oplus$z & (x$\oplus$y)(x$\oplus$z) \\ \cline{1-8}\rowcolor{lightgray}
              1 & 1 & 1 & 1  & \magenta{0}     & 0          & 0          & \magenta{0}              \\ \cline{1-8}\rowcolor{Cerulean}
              1 & 1 & 0 & 0  & \magenta{1}     & 0          & 1          & \magenta{0}              \\ \cline{1-8}\rowcolor{Cerulean}
              1 & 0 & 1 & 0  & \magenta{1}     & 1          & 0          & \magenta{0}              \\ \cline{1-8}
              1 & 0 & 0 & 0  & \magenta{1}     & 1          & 1          & \magenta{1}              \\ \cline{1-8}\rowcolor{lightgray}
              0 & 1 & 1 & 1  & \magenta{1}     & 1          & 1          & \magenta{1}              \\ \cline{1-8}
              0 & 1 & 0 & 0  & \magenta{0}     & 1          & 0          & \magenta{0}              \\ \cline{1-8}\rowcolor{lightgray}
              0 & 0 & 1 & 0  & \magenta{0}     & 0          & 1          & \magenta{0}              \\ \cline{1-8}
              0 & 0 & 0 & 0  & \magenta{0}     & 0          & 0          & \magenta{0}              \\ \cline{1-8}\rowcolor{lightgray}
            \end{tabular}
          }
        \end{center}
                Bueh, las líneas \blue{pintadas} muestran los contraejeplos
\end{enumerate}
