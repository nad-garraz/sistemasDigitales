\begin{enunciado}{\ejercicio}
  Una fórmula del álgebra de Boole es:
  \begin{itemize}[label=\tiny$\blacksquare$]
    \item $p,q, r,\dots$ una variable booleana que puede tener valor 1 o 0,
    \item 1, la constante \textit{verdadero},
    \item 0, la constante \textit{falso},
    \item Si $p$ y $q$ son fórmulas, entonces $p+q$ (p \texttt{OR} q),$pq$ ($p$ \texttt{AND} $q$) y $\bar{p}$ (la negación de $p$)
          son fórmulas.
  \end{itemize}
  ¿Se pueden expresar todas las funciones totales
  \footnote{Una función total es aquella para la que todo elemento del dominio tiene imagen.}
  $f: {0,1} \times {0,1} \to {0,1}$ usando fórmulas del álgebra de Boole? Justificar.
\end{enunciado}
