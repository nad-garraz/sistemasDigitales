\begin{enunciado}{\ejercicio}
  ¿Qué es \texttt{.text} y \texttt{.data}? ¿Qué tipo de información se guarda en cada una de ellas?
\end{enunciado}

Pongo un poco de contexto, para que no quede con sabor a nada la respuesta:

\bigskip

En el camino desde escribir el programa en \textit{alto nivel} (C, python, Java,etc.) hasta llegar al ejecutable, se pasa por el \textit{ensamblador}.

Las palabras \texttt{.text, .data, .word} entre otras son algunas de las \texttt{directivas del ensamblador} (\textit{assembler directives}).

\medskip

Cuando el programa
se carga en la memoria (para luego leerse) se genera un \textit{mapa de la memoria} el cual tiene una estructura particular dividida en \textit{segmentos},
donde se cargan cosas en cierto orden.

\medskip

Entre esos segmentos están el \texttt{Text Segment} y el \texttt{Global Data Segment}, segmentos a los que se accede con el \texttt{PC} (program counter) y el \texttt{GP} (global pointer) respectivamente.

\medskip

\texttt{The Text Segment}:
En esta parte de la memoria se guarda el código que escribimos en las instrucciones, \textit{la sustancia lógica que programamos en el Ripes o emulador favorito}.

\texttt{The Global Data Segment}:
En esta parte de la memoria se guardan variables globales. Éstas se pueden accedere desde cualquier parte
del código, desde adentro de alguna función por ejemplo. Esto sucede antes de que comience el programa.
\bigskip

Entonces cuando se usan las palabras \texttt{.text} y \texttt{.data}, lo hacemos para decirle al ensamblador en cual
segmento de la memoria meter el código que estamos escribiendo en el Ripes o emulador en cuestión.

% Contribuciones
\begin{aportes}
  %\aporte{url}{nombre icono}
  \item \aporte{\dirRepo}{Nad Garraz \github}
\end{aportes}
