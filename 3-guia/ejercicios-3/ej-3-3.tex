\begin{enunciado}{\ejercicio}
    Dado el siguiente arreglo de enteros de 16 bits en lenguaje Java:

    \begin{java}
        int [] arreglo16b = { -1 , 170 , 255 , -255 , 0 , 32 , 10000 , 0};
    \end{java}

    Sabiendo que este arreglo se guarda en memoria empezando en la direccion 0xCC.

    \begin{enumerate}
        \item Dibujar el estado de la memoria
        \item Si t0 = 0xCC, escribir un programa que dado un index i, devuelve \textbf{arreglo16b}[i]
    \end{enumerate}
    \end{enunciado}

\begin{enumerate}
\item
    \begin{center}
    \begin{tabular}{ |c|c|c|c|c|c|c|c|c|c|} 
        \hline
        Dirección & $\dots$ & 0xCC & 0xCD & 0xCE & 0xCF & 0xD0 & 0xD1 & 0xD2 & 0xD3   \\
        \hline       %        -1             170          255            -255         0             32  
        Valor     & $\dots$ & 0xFF & 0xFF & 0xAA & 0x00 & 0xFF & 0x00 & 0x01 & 0xFF   \\
        \hline
        
    \end{tabular}
    \end{center}

    \begin{center}
        \begin{tabular}{ |c|c|c|c|c|c|c|c|c|c|} 
            \hline
            Dirección & 0xD4 & 0xD5 & 0xD6 & 0xD7 & 0xD8 & 0xD9 & 0xDA & 0xDB & $\dots$ \\
            \hline
            Valor     & 0x00 & 0x00 & 0x30 & 0x00 & 0x10 & 0x27 & 0x00 & 0x00 &$\dots$ \\
            \hline
        \end{tabular}
    \end{center}

\item
    \begin{riscv}
        obtenerIndex:
        # Recibe en a0 la referencia de la posicion
        # Recibe en a1 la posicion a obtenerIndex
        # Devuelve en a0 el valor que tenia en esa posicion el array
        add a0, a0, a1
        lw a0, 0(a0)
    \end{riscv}
\end{enumerate}

    

    