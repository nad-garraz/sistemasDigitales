\begin{enunciado}{\ejercicio}
    \begin{enumerate}
        \item ¿Cuantos bytes ocupa cada instruccion de RISC-V? ¿Cual es la diferencia entre una
        instruccion y una pseudoinstruccion?

        \item ¿Que clases de instrucciones tiene la arquitectura RISC-V? ¿Que tipo de instrucciones 
        contiene cada clase? ¿Que diferencia hay entre instrucciones de Registros(\textbf{R}) y de Inmediatos
        (\textbf{I})?

        \item Ensamblar el siguiente codigo escrito en lenguaje RISC-V
        
        \begin{riscv}
    addi a6, x0, 10
    add a0, a1, a6
    bltz x1, 0x0ABC        
        \end{riscv}

        \item Desensamblar el siguiente programa escrito en lenguaje de maquina RISC-V.

        \begin{center}
            0111 1111 1111 0000 0000 0101 0001 0011 \\
            0101 0101 0101 0000 0000 0101 1001 0011 \\
            0000 0000 1010 0101 1100 0110 0011 0011 \\
            1111 1110 0000 0110 0000 1010 1110 0011 \\
            0000 0000 0000 0000 0000 0000 0001 0011 \\
        \end{center}
    \end{enumerate}
    \end{enunciado}

\begin{enumerate}
    \item 
        Cada instrucción de RISC V ocupa 4 bytes. \\
        La diferencia entre una instruccion y una pseudoinstruccion es que una pseudoinstruccion
        puede ser una o varias instrucciones, y al compilarlo a lenguaje máquina es convertido, 
        facilitando la programación, mientras que las instrucciones es lo que el procesador ejecuta en sí.
    
    \item

    \item

\end{enumerate}
    