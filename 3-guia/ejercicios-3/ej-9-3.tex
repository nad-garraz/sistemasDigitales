\begin{enunciado}{\ejercicio}
  Se cuenta con cuatro datos sin signo de 1-byte cada uno almacenados en el registro \texttt{t0} y queremos sumar el valor de los cuatro datos.
  Escribir un programa en lenguaje ensamblador RISC-V que realice esta operación y almacene el resultado en el registro \texttt{t0}.

  Ejemplo:
  \begin{center}
    { \tt
      \begin{tabular}{|c|c|c|c|c|}
        \hline
        t0 & 0x90 & 0x1A & 0x00 & 0x02 \\
        \hline
      \end{tabular}
    }
  \end{center}
  Con este dato el registro debería valer \texttt{0x000000AC}.
\end{enunciado}


\begin{riscv}
# Se usa la directiva de ensamblador .data
# se carga la palabra antes de comenzar 
# el programa

.data
    palabra: .byte 0x90
             .byte 0x1a
             .byte 0x00
             .byte 0x02

# Ahora se usa la directiva donde se usaran variable
# locales y ademas desde donde se podra acceder a 
# la informacion del segmento de memoria .data

.text
main:
    la s0 palabra   # s0 <- dirección del primer elemento de palabra 
    addi sp sp -4
    sw s0 0(sp)
    
    li a2 4 # contador inicializado en 4
    li t0 0 # acumulo la suma en t0
    
    for:
        lbu t1 0(s0) # t1 <-- valor del byte
        addi s0 s0 1 # Siguiente byte
        
        add t0 t0 t1 # suma
        
        addi a2 a2 -1 # actualizo el contador
        beqz a2 done
    j for
    
done:
    lw s0 4(sp)
    addi sp sp 4
    j fin
     
fin: j fin
\end{riscv}

