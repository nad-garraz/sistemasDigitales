\begin{enunciado}{\ejercicio}
    Dado el siguiente programa en lenguaje C.

    \begin{center}
    \begin{cLang}[linewidth=6cm]
    int x = 2;
    int y = 32;
    x = x + y ;
    \end{cLang}
\end{center}
    
    \begin{enumerate}
        \item Traducir a lenguaje esamblador de RISC-V. Usar los registros t0 y t1 inicilizados con
numeros de 8 bits para representar a las variables x e y respectivamente.
        \item  Escribir un programa que guarde en t2 un numero de 32 bits divido en sus 12 bits mas
        significativos en t0 y el resto de 20 bits en t1.
        \item Escribir un programa que guarde en t2 un numero de 32 bits divido en sus 12 bits mas
        significativos en t0 y el resto de 20 bits en \textbf{t1}.
    \end{enumerate}
    \end{enunciado}

    \begin{enumerate}
        \item 
        \begin{riscv}
    addi t0, zero, 2
    addi t1, zero, 32
    add t0, t0, t1
        \end{riscv}
        \item \hacer
        \item \hacer
    \end{enumerate}
    