%misc
\def\ytext{\quad \text{y} \quad}
\DeclareMathOperator{\X}{\text{x}}
\newcommand{\nBase}[2]{\texttt{#1}$_{\texttt{#2}}$}
\newcommand{\sigMag}{\texttt{signo+magnitud} }
\newcommand{\compDos}{\texttt{complemento a 2} }
\newcommand{\overflow}{\magenta{\texttt{overflow}} }

%% Cartelito HACER EJERCICIO
\newcommand{\hacer}{
        {\color{red!80!black}{\faIcon[regular]{flushed}... hay que hacerlo! \faIcon[regular]{sad-cry}}}\par
  {\color{black!70!white}
    \small Si querés mandarlo: Telegram $\to$ \href{\dirTelegram}{\small\faIcon{telegram}},
    o  mejor aún si querés subirlo en \LaTeX $\to$ \href{\dirRepo}{\small \faIcon{github}}.
  }\par
}


% separadores
\newcommand{\separador}{
  \par\noindent\rule{\linewidth}{0.4pt}\par
}
\newcommand{\separadorCorto}{
  \par\noindent\rule{0.5\linewidth}{0.4pt}\par
}

% Colores
\newcommand{\red}[1]{\textcolor{red}{#1}}
\newcommand{\green}[1]{\textcolor{OliveGreen}{#1}}
\newcommand{\blue}[1]{\textcolor{Cerulean}{#1}}
\newcommand{\cyan}[1]{\textcolor{cyan}{#1}}
\newcommand{\yellow}[1]{\textcolor{YellowOrange}{#1}}
\newcommand{\magenta}[1]{\textcolor{magenta}{#1}}
\newcommand{\purple}[1]{\textcolor{purple}{#1}}
\newcommand{\rosa}[1]{\textcolor{pink}{#1}}
\newcommand{\transparente}[1]{\color[rgb]{1,1,1,0.5}{#1}}

%llaves
\newcommand{\llave}[2]{ \left\{ \begin{array}{#1} #2 \end{array}\right. }

\newcommand{\ub}[2]{ \underbrace{\textstyle #1}_{\mathclap{#2}} }
\newcommand{\ob}[2]{ \overbrace{\textstyle #1}^{\mathclap{#2}} }

%=======================================================
% Comandos con flechas extensibles.
%=======================================================
% *Flechita* extensible con texto {arriba} y [abajo] 

\NewDocumentCommand{\flecha}{m o}{%
  \IfNoValueTF{#2}{%
    \xrightarrow[]{\text{#1}}
  }{
    \xrightarrow[\text{#2}]{\text{#1}}
  }
}


% Contributors
\newcommand{\aporte}[2]{
        \href{#1}{#2}
}

\newenvironment{aportes}{
\vspace{10pt}
\begin{minipage}{1\linewidth}
  \tt\footnotesize
  Los culpables de que esto haya sucedido:
\vspace{-10pt}
  \begin{multicols}{3}
  \begin{itemize}[label={\tiny\yellow{\faIcon{medal}}}]
      }{
    \end{itemize}
  \end{multicols}
\end{minipage}
}


% Update time
\def\update{\tiny
  {\today\ @ \currenttime}
}

% Dado que muchas veces ponemos cosas sobre un signo '='
%  acá está el comando para escribir \igual{arriba}[abajo] con texto!
\NewDocumentCommand{\igual}{m o}{
  \IfNoValueTF{#2}{
          \overset{\mathclap{\texttt{#1}}}{\texttt{=}}
  }{
          \tt \overset{\mathclap{\texttt{#1}}}{\underset{\mathclap{\texttt{#2}}}=}
  }
}

%=======================================================
% sección ejercicio con su respectivo formato y contador
%=======================================================
%iconito
\def\fueguito{{\color{orange}{\faIcon{fire}}}}

\newcounter{ejercicio}[section] % contador que se resetea en cada sección
\renewcommand{\theejercicio}{\arabic{ejercicio}} % el contador es un número arabic
\newcommand{\ejercicio}{%
  \stepcounter{ejercicio}% incremento en uno
  \titleformat{\section}[runin]{\bfseries}{\theejercicio}{1em}{}%
  \section*{Ejercicio \theejercicio}\labelEjercicio{ej:\theejercicio}
}

% Label y refencia para ejercicio hay alguna forma más elegante de hacer esto?
\newcommand{\labelEjercicio}[1]{
  \addtocounter{ejercicio}{-1} % counter - 1
  \refstepcounter{ejercicio} % referencia al anterior y luego + 1
  \label{#1}
  }
\newcommand{\refEjercicio}[1]{{\bf \ref{#1}.}}

%%%=======================================================
%%% fin sección ejercicio con su respectivo formato y contador
%%%=======================================================

\newenvironment{enunciado}[1]{% Toma un parametro obligatorio: \ejExtra o \ejercicio 
  \par
  \noindent
  \begin{minipage}{\linewidth}
    \separador % linea sobre el enunciado
    #1
    }% contenido
    {
    \separadorCorto % linea debajo del enunciado
    \par
  \end{minipage}\par
}
%%% (☞⌐▀͡ ͜ʖ͡▀ )☞ Yo mama
\def\neverGonnaGiveYouUp{https://www.youtube.com/watch?v=dQw4w9WgXcQ}
\def\zanguango{https://www.youtube.com/watch?v=Uzcl2gNL3zg&t=10s}
\def\justDoIt{https://www.youtube.com/watch?v=ZXsQAXx_ao0}
\def\ariane5{https://youtu.be/PK_yguLapgA?t=91}
\def\chinito{https://www.youtube.com/watch?v=ebz4xuPf-is}

%%% Iconos más usados
\def\github{\faIcon{github}}
\def\instagram{\faIcon{instagram}}
\def\tiktok{\faIcon{tiktok}}
\def\linkedin{\faIcon{linkedin}}
