\documentclass[12pt,a4paper, spanish, twoside]{article}

\usepackage[headheight=110pt, top = 2cm, bottom = 2cm, left=1cm, right=1cm]{geometry} %modifico márgenes
\usepackage[T1]{fontenc} % tildes
\usepackage[utf8]{inputenc} % Para poder escribir con tildes en el editor.
\usepackage[english]{babel} % Para cortar las palabras en silabas, creo.
\usepackage[ddmmyyyy]{datetime}
\usepackage{titlesec} % para editar titulos y hacer secciones con formato a medida
\usepackage{amsmath} % Soporte de mathmatics
\usepackage{amssymb} % fuentes de mathmatics
\usepackage{mathtools} % Más herramientas para matemáctica
\usepackage{array} % Para tablas y eso
\usepackage{caption} % Configuracion de figuras y tablas
\usepackage[dvipsnames]{xcolor} % Para colorear el texto: black, blue, brown, cyan, darkgray, gray, green, lightgray, lime, magenta, olive, orange, pink, purple, red, teal, violet, white, yellow.
\usepackage{graphicx} % Necesario para poner imagenes
\usepackage{enumitem} % Cambiar labels y más flexibilidad para el enumerate
\usepackage{multicol}
\usepackage{cancel}
\usepackage{skull} % símbolos de donde uso Skull
\usepackage{tikz} % para graficar
\usepackage{fontawesome5} % fuentes "extras"
\usepackage{bbding} % símbolos de donde uso FiveStar
\usepackage{lipsum} % dummy text
\usepackage{qrcode} % genera útiles qr
\usepackage{circuitikz} % para poner circuitos
\usepackage{colortbl} % mejora visibilidad de tablas
\usepackage{listings} % Escribir código
\usepackage{color} % Poner color a codigos
\usepackage{mdframed} %nice frames
\usepackage{paracol} % To have columns that goes in more than 1 page


% para hacer los graficos tipo grafos
\usetikzlibrary{shapes,arrows.meta, chains, matrix, calc, trees, positioning, fit}
\usetikzlibrary{external}


\definecolor{lbcolor}{rgb}{0.9,0.9,0.9}  

% language definition
\lstdefinelanguage[RISC-V]{Assembler}
{
  alsoletter={.}, % allow dots in keywords
  alsodigit={0x}, % hex numbers are numbers too!
  morekeywords=[1]{ % instructions
    lb, lh, lw, lbu, lhu,
    sb, sh, sw,
    sll, slli, srl, srli, sra, srai,
    add, addi, sub, lui, auipc,
    xor, xori, or, ori, and, andi,
    slt, slti, sltu, sltiu,
    beq, bne, blt, bge, bltu, bgeu,
    j, jr, jal, jalr, ret,
    scall, break, nop
  },
  morekeywords=[2]{ % sections of our code and other directives
    .align, .ascii, .asciiz, .byte, .data, .double, .extern,
    .float, .globl, .half, .kdata, .ktext, .set, .space, .text, .word
  },
  morekeywords=[3]{ % registers
    zero, ra, sp, gp, tp, s0, fp,
    t0, t1, t2, t3, t4, t5, t6,
    s1, s2, s3, s4, s5, s6, s7, s8, s9, s10, s11,
    a0, a1, a2, a3, a4, a5, a6, a7,
    ft0, ft1, ft2, ft3, ft4, ft5, ft6, ft7,
    fs0, fs1, fs2, fs3, fs4, fs5, fs6, fs7, fs8, fs9, fs10, fs11,
    fa0, fa1, fa2, fa3, fa4, fa5, fa6, fa7
  },
  morecomment=[l]{;},   % mark ; as line comment start
  morecomment=[l]{\#},  % as well as # (even though it is unconventional)
  morestring=[b]",      % mark " as string start/end
  morestring=[b]'       % also mark ' as string start/end
}

\lstnewenvironment{riscv}
{
    \lstset{language={[RISC-V]Assembler}, 
    backgroundcolor=\color{lbcolor},
    numbers=left}}
{}

\lstnewenvironment{java}
{
    \lstset{language=java, 
    backgroundcolor=\color{lbcolor},
    numbers=left}}
{} %Import the languages stored in languages.tex

\usepackage{fancyhdr} % Encabezados y pie de páginas

% En general quiero que este paquete sea el último en importarse
\usepackage{hyperref} % para que haya links navegables en el PDF
\hypersetup{
  colorlinks=true,
  linkcolor=blue,
  %filecolor=magenta,
  urlcolor=OliveGreen!90!black,
  pdftitle={Sistemas Digitales 2C 2024},
  pdfauthor={NadGarraz y comunidad}
}
\urlstyle{same}

\setlength{\parindent}{0pt} % Para que no haya indentación en las nuevas líneas.

%% Info SOCIAL
\def\dirRepo{https://github.com/nad-garraz/sistemasDigitales}
\def\dirTelegram{https://t.me/joinchat/DS9ZukGbZgIOIaHgdBlavQ}
\newcommand{\dirGuia}[1]{\dirRepo/blob/main/#1-guia/#1-sol.pdf}


