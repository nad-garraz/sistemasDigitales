\documentclass[12pt,a4paper, spanish, twoside]{article}

\usepackage[headheight=110pt, top = 2cm, bottom = 2cm, left=1cm, right=1cm]{geometry} %modifico márgenes
\usepackage[T1]{fontenc} % tildes
\usepackage[utf8]{inputenc} % Para poder escribir con tildes en el editor.
\usepackage[english]{babel} % Para cortar las palabras en silabas, creo.
\usepackage[ddmmyyyy]{datetime}
\usepackage{titlesec} % para editar titulos y hacer secciones con formato a medida
\usepackage{amsmath} % Soporte de mathmatics
\usepackage{mathtools} % Más herramientas para matemáctica
\usepackage{amssymb} % fuentes de mathmatics
\usepackage{array} % Para tablas y eso
\usepackage{caption} % Configuracion de figuras y tablas
\usepackage[dvipsnames]{xcolor} % Para colorear el texto: black, blue, brown, cyan, darkgray, gray, green, lightgray, lime, magenta, olive, orange, pink, purple, red, teal, violet, white, yellow.
\usepackage{graphicx} % Necesario para poner imagenes
\usepackage{enumitem} % Cambiar labels y más flexibilidad para el enumerate
\usepackage{multicol}
\usepackage{cancel}
\usepackage{skull} % símbolos de donde uso Skull
\usepackage{tikz} % para graficar
\usepackage{fontawesome5} % fuentes "extras"
\usepackage{bbding} % símbolos de donde uso FiveStar
\usepackage{lipsum} % símbolos de donde uso FiveStar
\usepackage{qrcode}

% para hacer los graficos tipo grafos
\usetikzlibrary{shapes,arrows.meta, chains, matrix, calc, trees, positioning, fit}
\usetikzlibrary{external}

\usepackage{fancyhdr} % Encabezados y pie de páginas

% En general quiero que este paquete sea el último en importarse
\usepackage{hyperref} % para que haya links navegables en el PDF
\hypersetup{
  colorlinks=true,
  linkcolor=blue,
  %filecolor=magenta,
  urlcolor=OliveGreen!90!black,
  pdftitle={Sistemas Digitales 2C 2024},
  pdfauthor={NadGarraz y comunidad}
}
\urlstyle{same}

\setlength{\parindent}{0pt} % Para que no haya indentación en las nuevas líneas.

%% Info SOCIAL
\def\dirRepo{https://github.com/nad-garraz/sistemasDigitales}
\def\dirTelegram{https://t.me/+X4p0xKnXp0Y3ZThh}
\newcommand{\dirGuia}[1]{\dirRepo/blob/main/#1-guia/#1-sol.pdf}
