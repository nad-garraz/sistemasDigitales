%%% ===========================
%%% Iconos para los itemize
\def\iconTeoriaUno{\tiny\faIcon{laptop}}
\def\iconTeoriaDos{\tiny\faIcon{calculator}}
\def\iconTeoriaTres{\tiny\faIcon{laptop-code}}
%%% ===========================

\begin{itemize}[label=\iconTeoriaUno]
  \item  \textit{Complejidad:}\par
        Es lo que nos da la necesida de abstraer.

  \item \textit{Abstracción}: \par
        Para lidiar con la \textit{complejidad} que tienen los sistemas usamos la abstracción.\par
        De \underline{menor} a \underline{mayor} abstracción:
        \begin{center}
          electrones
          $\to$
          transistores
          $\to$
          circuitos analógicos
          $\to$
          circuitos digitales
          $\to$
          circuitos de lógica
          $\to$
          micro-arquitectura
          $\to$
          arquitectura
          $\to$
          sistema operativo
          $\to$
          aplicaciones de software
        \end{center}

        Lo más abstracto es más fácil de controlar e implementar que lo menos abstracto.

  \item \textit{Sistemas numéricos:}
        \begin{itemize}[label=\iconTeoriaDos]
          \item \textit{Decimal:}
                \begin{center}
                  \nBase{9745}{10} $ = 9 \times 10^3 + 7 \times 10^2 + 4 \times 10^1 + 5 \times 10^0$
                \end{center}
                El rango de un número \underline{\textit{decimal}} con \texttt{n} dígitos mayor o igual a cero es $10^{\texttt{n}}: 0,1, \dots, 10^{\texttt{n}} - 1$

          \item \textit{Binario:}
                \begin{center}
                  \nBase{1101}{2} $= 1 \times 2^3 + 1 \times 2^2 + 0 \times 10^1 + 1 \times 10^0 =$ \nBase{13}{10}.
                \end{center}
                El rango de un número \underline{\textit{binario}} con \texttt{n} dígitos mayor o igual a cero es $2^{\texttt{n}}: 0,1, \dots, 2^{\texttt{n}} - 1$.
                Si tengo solo un bit, \textit{\red{b}inary dig\red{it}}, puedo obtener o bien 0 o bien 1. Si tengo dos bits, entonces
                puedo tener $2^2$ dígitos, 00, 01, 10, 11. (el rango es después de todo 4).

          \item \textit{Hexadecimal:}
                \begin{center}
                  \nBase{2AD}{16} $= 2 \times 16^2 + \texttt{A} \times 16^1 + \texttt{D} \times 16^0 =$ \nBase{685}{10}
                \end{center}
                El rango de un número \underline{\textit{hexadecimal}} con \texttt{n} dígitos mayor o igual a cero es $16^{\texttt{n}}: 0,1, \dots, 16^{\texttt{n}} - 1$.
                Cada dígito de un número en base hexadecimal corresponde a un número binario
                de 4-bits. Después de todo un número $i_{16}$ de \textit{un solo dígito} tiene un
                rango de 16: $0,\dots 15$ y
                un número binario de 4-bits tiene un rango de $2^4 = 16$ \checkmark

                \begin{center}
                  \nBase{2AD}{16} $= \ub{\texttt{0010}}{\texttt{2}} \ub{\texttt{1010}}{\texttt{A}}\ub{\texttt{1101}}{\texttt{D}}\,_{\texttt{2}}$
                \end{center}

        \end{itemize}

  \item \textit{Operaciones entre números binarios:} Se poné picante según la \underline{representación usada}.
        \begin{itemize}[label=\iconTeoriaDos]
          \item Sumar es fácil si los números son positivos. Tengo \overflow si el resultado
                tiene más cifras que bits disponibles para almacenar dicho resultado.
        \end{itemize}

  \item \textit{Números binarios con signo: }\par
        Hay distintas \textit{representaciones} para hacer esto, acá están las 2 más usadas:
        \begin{itemize}[label=\iconTeoriaDos]

          \item \sigMag\par
                El bit más significativo, el de más a la izquierda, marca el \purple{signo}.
                Un número con \texttt{n} bits en esta representación tiene un rango: $[-2^{\texttt{n}-1}+1, 2^{\texttt{n}-1} - 1]$.
                Por ejemplo con 3-bits:\par
                El rango es [-3, 3]
                $$
                  \begin{array}{|ccc|}
                    \hline
                    \texttt{Base 2}       & \to & \texttt{Base 10}  \\
                    \hline
                    \texttt{\purple{0}00} & \to & \texttt{\red{0}}  \\
                    \texttt{\purple{0}01} & \to & \texttt{1}        \\
                    \texttt{\purple{0}10} & \to & \texttt{2}        \\
                    \texttt{\purple{0}11} & \to & \texttt{3}        \\
                    \texttt{\purple{1}00} & \to & \texttt{\red{-0}} \\
                    \texttt{\purple{1}01} & \to & \texttt{-1}       \\
                    \texttt{\purple{1}10} & \to & \texttt{-2}       \\
                    \texttt{\purple{1}11} & \to & \texttt{-3}       \\
                    \hline
                  \end{array}
                $$
                \textit{Sumar normalmente en esta representación no tiene sentido}. Representa en total 2$^\texttt{n} - 1$ elementos
                porque el \red{-0} está usando un lugar al pedo.

          \item \hypertarget{teoria-1:complementoA2}{\compDos:}\par
                \begin{itemize}[label=\iconTeoriaTres]
                  \item Menos intuitivo, pero más útil. Si tengo un número de \texttt{n}-bits, voy a tener siempre:
                        \begin{center}
                          \begin{tabular}{|rcr|l}
                            \cline{1-3}
                            \nBase{0}{10}                      & $\to$ & \nBase{00$\dots$00}{2}  &                                      \\
                            \nBase{-1}{10}                     & $\to$ & \nBase{11$\dots$11}{2}  &                                      \\
                            \nBase{$-2^{\texttt{n}-1}$}{10}    & $\to$ & \nBase{100$\dots$00}{2} & $\to$ \textit{\purple{weird number}} \\
                            \nBase{$2^{\texttt{n}-1} - 1$}{10} & $\to$ & \nBase{011$\dots$11}{2} &                                      \\
                            \cline{1-3}
                          \end{tabular}
                        \end{center}
                        Teniendo al 0, $-1, \,-2^{\texttt{n}-1} \ytext (2^{
                            \texttt{n}-1} -1)$, puedo encontrar el resto de la \textit{representación}:

                  \item\hypertarget{teoria-1:weirdNumber}{} Para encontrar el opuesto a un número, se cambian los 0 por 1 y bicerveza
                        \footnote{chiste: bicerveza = \faIcon{beer}\faIcon{beer}. Se escribe viceversa {\tiny\faIcon{hand-middle-finger}\faIcon{angry}}.}
                        , luego se le suma 1 a eso, ej:
                        \begin{center}
                          \nBase{5}{10} = \nBase{0101}{2}
                          $\flecha{busco}[$-5$]$
                          \nBase{1010}{2} + \nBase{0001}{2} = \nBase{1011}{2} = \nBase{-5}{10}.
                        \end{center}
                        Cosa que no funciona con el \purple{\textit{weird number}}, porque su complemento te da a él mismo \href{\neverGonnaGiveYouUp}{\faIcon{hat-wizard}}.

                  \item El rango es de $[-2^{\texttt{n}-1}, 2^{\texttt{n}-1} -1],\,2^{\texttt{n}}$ elementos. Hay un elemento negativo más que positivos.

                  \item Ejemplito: En 3-bits me encuentro todo el conjunto, $[-4, 3]$:\par
                        \begin{center}
                          \begin{tabular}{|rcr|}
                            \hline
                            \nBase{000}{2} & $\to$ & \nBase{0}{10}  \\
                            \nBase{001}{2} & $\to$ & \nBase{1}{10}  \\
                            \nBase{010}{2} & $\to$ & \nBase{2}{10}  \\
                            \nBase{011}{2} & $\to$ & \nBase{3}{10}  \\
                            \nBase{100}{2} & $\to$ & \nBase{-4}{10} \\
                            \nBase{101}{2} & $\to$ & \nBase{-3}{10} \\
                            \nBase{110}{2} & $\to$ & \nBase{-2}{10} \\
                            \nBase{111}{2} & $\to$ & \nBase{-1}{10} \\
                            \hline
                          \end{tabular}
                        \end{center}

                  \item \textit{Suma en \compDos:}
                        Si sumo dos números de distinto signo no voy a tener \overflow!
                        \begin{center}
                          4 bits:
                          \nBase{-4}{10} + \nBase{5}{10} =
                          \nBase{1100}{2} + \nBase{0101}{2} =
                          \nBase{$\cancel{\texttt{1}}$0001}{2} = \nBase{1}{10}\par
                        \end{center}

                  \item \textit{Sign extension:}
                        Para encontrar la representación de un número conocido con más bits.\par
                        Copio el signo al resto de los númerosengo que mandar el bit del signo
                        hacia el dígito más significativo, :
                        \begin{center}
                          {\tt 4 bits: \nBase{6}{10} = \nBase{0110}{2} y \nBase{-5}{10} = \nBase{1011}{2}}\par
                          Extendido a 8 bits: \nBase{6}{10} = \nBase{(0000 0110)}{2} y \nBase{-5}{10} = \nBase{(1111 1011)}{2}\par
                        \end{center}
                \end{itemize}
        \end{itemize}
  \item \texttt{Exceso m:}
        \begin{itemize}[label=\iconTeoriaTres]

          \item Se desplaza el \texttt{0} a la posición \texttt{m}. Es así que si la representación es
                \texttt{exceso 4} en \texttt{6-bits} base 3: \nBase{0}{10} = \nBase{(000 0011)}{3}
        \end{itemize}

  \item \textit{Comparación representaciones del mismo} \magenta{\texttt{Dato}}:\par
        \ttfamily\begin{tabular}{|c||c|c|c|c|c|}
          \hline
          Posición & \magenta{Dato}    & unsigned  & \sigMag   & Exceso m (m=4) & \compDos  \\ \hline \hline
          0        & \nBase{(0000)}{2} & \blue{0}  & \red{0}   & -4             & \blue{0}  \\ \hline
          1        & \nBase{(0001)}{2} & 1         & 1         & -3             & 1         \\ \hline
          2        & \nBase{(0010)}{2} & 2         & 2         & -2             & 2         \\ \hline
          3        & \nBase{(0011)}{2} & 3         & 3         & -1             & 3         \\ \hline
          4        & \nBase{(0100)}{2} & 4         & 4         & \yellow{0}     & 4         \\ \hline
          5        & \nBase{(0101)}{2} & 5         & 5         & 1              & 5         \\ \hline
          6        & \nBase{(0110)}{2} & 6         & 6         & 2              & 6         \\ \hline
          7        & \nBase{(0111)}{2} & 7         & \blue{7}  & 3              & \blue{7}  \\ \hline
          8        & \nBase{(1000)}{2} & 8         & \red{-0}  & 4              & \blue{-8} \\ \hline
          9        & \nBase{(1001)}{2} & 9         & -1        & 5              & -7        \\ \hline
          10       & \nBase{(1010)}{2} & 10        & -2        & 6              & -6        \\ \hline
          11       & \nBase{(1011)}{2} & 11        & -3        & 7              & -5        \\ \hline
          12       & \nBase{(1100)}{2} & 12        & -4        & 8              & -4        \\ \hline
          13       & \nBase{(1101)}{2} & 13        & -5        & 9              & -3        \\ \hline
          14       & \nBase{(1110)}{2} & 14        & -6        & 10             & -2        \\ \hline
          15       & \nBase{(1111)}{2} & \blue{15} & \blue{-7} & \blue{11}      & -1        \\ \hline
        \end{tabular}
\end{itemize}

% Contribuciones
\begin{aportes}
  %\aporte{url}{nombre icono}
  \item \aporte{\dirRepo}{Nad Garraz \github}
\end{aportes}
