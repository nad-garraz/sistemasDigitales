\begin{enunciado}{\ejercicio}
  ¿Puede alguna cadena binaria de \texttt{k} dígitos, interpretada en \compDos,
  representar un número que no puede ser representado por una cadena de la misma longitud, interpretada en \sigMag?
  ¿Y al revés?
\end{enunciado}

\hyperlink{teoria-1:complementoA2}{De al teoría de \compDos}, tengo que el rango es distinto en las distintas interpretaciones. Hay
un número más en la de \compDos.\par

Si tengo \texttt{k} dígitos 
\nBase{(-2$^{\texttt{k-1}}$)}{10} \texttt{=} \nBase{($\ub{\mathtt{100...0}}{\mathtt{k-bits}}$)\,}{2}.
Número que no puedo representar en \sigMag porque el menor número es \nBase{(-2$^{\texttt{k-1}}$ + 1)}{10}.\par

La representación \compDos contiene a todos los número de la representación \sigMag.

\red{¿Hay algo que aprender? O es solo eso?}
