\begin{enunciado}{\ejercicio}
  %Enunciado
  Dar ocho pares de numeros tales que la suma de las representaciones de cada
  par en complemento a dos de 4 bits provoque lo siguiente:

  \begin{enumerate}[label=\arabic*)]
    \item No se produzca acarreo ni \overflow.
    \item Se produzca acarreo pero no \overflow.
    \item Se produzca acarreo y \overflow.
    \item No se produzca acarreo pero si \overflow.
    \item Se produzca acarreo y el resultado sea cero.
    \item No se produzca acarreo y el resultado sea cero.
    \item El resultado sea negativo y se produzca \overflow.
    \item El resultado sea negativo y no se produzca \overflow.
  \end{enumerate}
\end{enunciado}

\begin{enumerate}[label=\arabic*)]
  \item \nBase{(0000, 0000)}{2} o \nBase{(0101, 1010)}{2}
  \item \nBase{(0001, 0001)}{2}
  \item\label{ej-9-1:item3} \nBase{(1111, 1111)}{2}
  \item \nBase{(1000, 1111)}{2}
  \item \nBase{(0001, 1111)}{2}
  \item \nBase{(0000, 0000)}{2}
  \item\label{ej-9-1:item7} \nBase{(1111, 1111)}{2}
  \item \nBase{(1111, 0000)}{2} o \nBase{(1111, 1001)}{2} o \nBase{(1111, 1100)}{2}

        Nota: Los incisos \ref{ej-9-1:item3} y \ref{ej-9-1:item7} están mal. Los dejamos ahí para
        testearte \href{\ariane5}{y hacerte saber que lograste esto \faIcon[solid]{handshake}}.\par

        Cuando usamos la representación que usemos, el \overflow tiene que ver con el sentido que le damos
        a cierto número. Por ejemplo:\par

        \begin{tabular}{rcr}
          \texttt{Acarreo} &     & \nBase{\purple{111111\ }}{\ }        \\
                           &     & \nBase{100001}{2}                    \\
                           & $+$ & \nBase{011111}{2}                    \\ \cline{2-3}
                           &     & \nBase{\purple{1}00000\purple{0}}{2}
        \end{tabular}
        Acabo de hacer \nBase{-7}{10} + \nBase{7}{10} = \nBase{0}{10}, el \texttt{bit} más significativo que supera la cantidad de
        bits de mi cuenta se descarta y el resultado es coherente con mi represntación. No hay \overflow.
\end{enumerate}

% Contribuciones
\begin{aportes}
  %\aporte{url}{nombre icono}
  \item \aporte{https://github.com/misProyectosPropios}{Iñaki Frutos \github}
  \item \aporte{\dirRepo}{Nad Garraz \github}
\end{aportes}
