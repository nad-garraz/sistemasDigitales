\begin{enunciado}{\ejercicio}
  ¿Puede suceder en alguna base que la suma de dos números de precisión fija
  tenga un acarreo mayor que 1?
  Exhibir un ejemplo o demostrar lo contrario.
\end{enunciado}

En cualquier base \texttt{B} se da que el último símbolo que puede reprentar es
\nBase{B-1}{}. Entonces cuando se sumen 2 dígitos:
\begin{center}
  \tt
  \begin{tabular}{r}
    B - 1 \\
    B - 1 \\ \hline
    2B -2
  \end{tabular}
\end{center}

La expresión de un número en base \texttt{B} que tiene un 2 de peso:
\begin{center}
  \tt
  x = 2 B$^\mathtt{n+1}$,
\end{center}
y otro que tiene como peso a la suma de los mayores pesos de la base:
\begin{center}
  \tt
  y = (2B - 2) B$^\mathtt{n}$
\end{center}
Pero, pero, pero:
\begin{center}
  \tt
  2 B$^\mathtt{n+1}$ $\stackrel{\mathtt{?}}{\geq}$ (2B - 2) B$^\mathtt{n}$\par
  0 $\stackrel{\mathtt{\checkmark}}{\geq}$  - 2 B$^\mathtt{n}$
\end{center}

Por lo que se estaría complicando para el \texttt{team acarreo mayor a 1}

% Contribuciones
\begin{aportes}
  %\aporte{url}{nombre icono}
  \item \aporte{\dirRepo}{Nad Garraz \github}
\end{aportes}
