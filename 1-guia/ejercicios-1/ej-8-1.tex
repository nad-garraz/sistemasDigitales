\begin{enunciado}
  {\ejercicio} ¿Cómo acomodaría esta suma de números \texttt{hexadecimales} de 4
  dígitos en notación \compDos, para que en ningún momento se produzca \overflow?
  \begin{center}
    \tt \nBase{7744}{16} + \nBase{5499}{16} + \nBase{6788}{16} + \nBase{AB68}{16}
    + \nBase{88BD}{16} + \nBase{9878}{16} = \nBase{0003}{16}
  \end{center}
\end{enunciado}

Dado que sabemos la suma de un número positivo y un número negativo nunca puede resultar
en \overflow, planteemos 3 parejas de números con uno positivo y otro negativo.
\newline
\newline
No hace falta pasar a binario para poder fijarse el signo de cada uno. En un
número hexadecimal, si el digito más significativo está en el rango $[8;F]$ podemos
asegurar que es un número negativo. Caso contrario será un número positivo.
\begin{center}
  \begin{multicols}{3}
    \begin{tabular}{rl}
          & \nBase{7744}{16} \\
      $+$ & \nBase{88BD}{16} \\
      \hline
          & \nBase{0001}{16}
    \end{tabular}

    \begin{tabular}{rl}
          & \nBase{5499}{16} \\
      $+$ & \nBase{AB68}{16} \\
      \hline
          & \nBase{0001}{16}
    \end{tabular}

    \begin{tabular}{rl}
          & \nBase{6788}{16} \\
      $+$ & \nBase{9878}{16} \\
      \hline
          & \nBase{0001}{16}
    \end{tabular}
  \end{multicols}
  Y bueno por último...
  \begin{center}
    $0001_{16}+0001_{16}+0001_{16}= 0003_{16}$
  \end{center}
\end{center}

% Contribuciones
\begin{aportes}
  %\aporte{url}{nombre icono}
  \item \aporte{https://github.com/rorofino10}{Román O. \github}
\end{aportes}
