\begin{enunciado}{\ejercicio}
  Diremos que un sistema de representación de números como
  cadenas binarias de longitud fija es \texttt{biyectivo} si no admite
  más de una representación para cada número y toda cadena disponible
  es utilizada para representar algún número.\par
  Decidir si la siguiente afirmación es verdadera o falsa:
  \begin{center}
    "No es posible dar con un sistema que represente números
    con signo utilizando cadenas binarias de longitud fija que sea
    \texttt{biyectivo}, tenga una representación para el cero y donde
    la cantidad de números positivos y negativos representados sea la misma".
    Justificar.
  \end{center}
\end{enunciado}

Esto buscando algo de esta pinta:
\begin{center}
  \ttfamily\begin{tabular}{|c|c|c|c|}
    \hline
    Posición & \magenta{Dato}   & casi 1 & casi 2 \\ \hline \hline
    0        & \nBase{(000)}{2} & -3     & -4     \\ \hline
    1        & \nBase{(001)}{2} & -2     & -3     \\ \hline
    2        & \nBase{(010)}{2} & -1     & -2     \\ \hline
    3        & \nBase{(011)}{2} & 0      & -1     \\ \hline
    4        & \nBase{(100)}{2} & 1      & 0      \\ \hline
    5        & \nBase{(101)}{2} & 2      & 1      \\ \hline
    6        & \nBase{(110)}{2} & 3      & 2      \\ \hline
    7        & \nBase{(111)}{2} & 4      & 3      \\ \hline
  \end{tabular}
\end{center}

Y está complicado porque tengo que usar de $\mathtt{2}^\mathtt{n}$ elementos
Uno para el cero, por lo que quedan:

\begin{center}
  {\tt $\mathtt{2}^{\mathtt{n}}$ = 2$^\mathtt{n}$ - 1 $\to$ 2$^\mathtt{n}$ - 1 mod 2 $\not=$ 0}
\end{center}
luego no podría tener igual cantidad de números positivos y negativos.

% Contribuciones
\begin{aportes}
  %\aporte{url}{nombre icono}
  \item \aporte{\dirRepo}{Nad Garraz \github}
\end{aportes}
