\begin{enunciado}{\ejercicio}
  Sean los siguientes numerales binarios de ocho dígitos:\par
  \begin{center}
    {\tt
      r = \nBase{(1011 1111)}{2},
      s = \nBase{(1000 0000)}{2} y
      t = \nBase{(1111 1111)}{2}.
    }
  \end{center}
  \par
  ¿Qué números representan si asumimos que son codificaciones de
  enteros en complemento a 2? ¿ Y si fueran codificaciones en signo+magnitud?
\end{enunciado}

En la
\hyperlink{teoria-1:complementoA2}{teoría están cuales son los valores \textit{amigos} de la representación complemento a 2}
\begin{itemize}[label=\tiny\faIcon{smile}]
  \item
        Notar que si \texttt{r} es de complemento:\par
        {\tt
          r + 64 = \nBase{(1011 1111)}{2} + \nBase{(0100 0000)}{2} =
          \nBase{(1111 1111)}{2} = \nBase{-1}{10}
          $\to$
                r = \nBase{-65}{10}
        }\par
        Puedo llegar a lo mismo usando la técnica para encontrar el complemento de \texttt{r}:\par
        $\overline{\mathtt{r}} = \overline{\mathtt{1011 1111}}_{\mathtt{2}}$
        $\flecha{cambio}[\texttt{1}$\leftrightarrow$\texttt{0}]$
        \nBase{(0100 0000)}{2}
        $\flecha{sumo}[1]$
        \nBase{(0100 000\blue{1})}{2} =
                \nBase{}{2} = \nBase{65}{10} = $\overline{\mathtt{r}}$, entonces \texttt{r} = \nBase{-65}{10}

        Si fuese de signo+magnitud:
        {\tt r = \nBase{(1011 1111)}{2} = \nBase{-63}{10} }\par
        Sumar normalmente en la representación de signo+magnitud es para problemas.

  \item
        Para complemento a 2: {\tt s = \nBase{(1000 0000)}{2} = \nBase{-128}{10} }
        es el \purple{\hyperlink{teoria-1:weirdNumber}{\textit{weird number}}}\par
        Y en signo+magnitud: {\tt s = \nBase{(1000 0000)}{2} = \nBase{-0}{10} }

  \item
        Para complemento a 2: {\tt t = \nBase{(1111 1111)}{2} = \nBase{-1}{10} } \par
        Y en signo+magnitud: {\tt t = \nBase{(1111 1111)}{2} = \nBase{-127}{10} }

\end{itemize}

