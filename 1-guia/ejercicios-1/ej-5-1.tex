\begin{enunciado}{\ejercicio}
  Codificar los siguientes números en base 2, usando la precisión y forma de representación
  indicada en cada caso. Comparar los resultados.

  \begin{enumerate}[label=\tiny $\blacksquare_{\tt\alph*}$]
    \item \nBase{0}{10} $\to$ usando 8 bits, notación \sigMag y notación \compDos.
    \item \nBase{-1}{10} $\to$ usando 8 y 16 bits, en ambos casos notación \sigMag y notación \compDos.
    \item \nBase{255}{10} $\to$ usando 8 bits notación sin signo y 16 bits notación \compDos.
    \item \nBase{-128}{10} $\to$ usando 8 y 16 bits, en ambos casos notación \compDos.
    \item \nBase{128}{10} $\to$ usando 8 bits, notación sin signo y 16 notación \compDos.
  \end{enumerate}
\end{enunciado}
Leer la \hyperlink{teoria-1:complementoA2}{técnica para encontrar los números con \compDos}.
{
\tt
\begin{enumerate}[label=\tiny $\blacksquare_{\tt\alph*}$]
  \item
        \sigMag: \nBase{0}{10} = \nBase{(0000 0000)}{2} $\igual{\red{?}}$ \nBase{(1000 0000)}{2}. \red{abuso de notación?}\par

        \compDos: \nBase{0}{10} = \nBase{(0000 0000)}{2}.

  \item
        \sigMag:\par
        \nBase{-1}{10} = \nBase{(1000 0001)}{2}.\par
        \nBase{-1}{10} = \nBase{(1000 0000 0000 0001)}{2}.\par

        \compDos:\par
        \nBase{-1}{10} = \nBase{(1111 1111)}{2}.\par
        \nBase{-1}{10} = \nBase{(1111 1111 1111 1111)}{2}.\par

  \item
        Sin Signo:\par
        \nBase{255}{10} = \nBase{(1111 1111)}{2}.\par
        \compDos:\par
        \nBase{255}{10} = \nBase{(0000 0000 1111 1111)}{2}.\par

  \item
        \compDos:\par
        \nBase{-128}{10} = \nBase{(1000 0000)}{2}.\par
        \nBase{-128}{10} $\mathtt{\to}$ \nBase{128}{10} = \nBase{(0000 0000 1000 0000)}{2}
        $\flecha{busco}[opuesto]$
        \nBase{(1111 1111 1000 0000)}{2} = \nBase{-128}{10}.\par

  \item
        Sin Signo:\par
        \nBase{128}{10} = \nBase{(1000 0000)}{2}.\par
        \nBase{128}{10} = \nBase{(0000 0000 1000 0000)}{2}.\par
\end{enumerate}
\red{¿Qué se puede interpretar de esto?}
}

\begin{aportes}
  %\aporte{url}{nombre icono}
  \item \aporte{\dirRepo}{Nad Garraz \github}
\end{aportes}
