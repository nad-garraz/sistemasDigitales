\begin{enunciado}{\ejercicio}
  Dar un ejemplo de un sistema de representación \texttt{biyectivo} en
  el que lal cantidad de números positivos y negativos representados es la misma.
\end{enunciado}

\texttt{Exceso 4: 2-trits}. Como tengo cantidad de impar de elementos que representar \href{\neverGonnaGiveYouUp}{es así}.

\begin{center}
  \tt\begin{tabular}{|c|c|c|}
    \hline
    Posición & \magenta{Dato}  & Exceso 4     \\ \hline \hline
    0        & \nBase{(00)}{3} & \magenta{-4} \\ \hline
    1        & \nBase{(01)}{3} & \magenta{-3} \\ \hline
    2        & \nBase{(02)}{3} & \magenta{-2} \\ \hline
    3        & \nBase{(10)}{3} & \magenta{-1} \\ \hline
    4        & \nBase{(11)}{3} & \blue{0}     \\ \hline
    5        & \nBase{(12)}{3} & \yellow{1}   \\ \hline
    6        & \nBase{(20)}{3} & \yellow{2}   \\ \hline
    7        & \nBase{(21)}{3} & \yellow{3}   \\ \hline
    8        & \nBase{(22)}{3} & \yellow{4}   \\ \hline
  \end{tabular}
\end{center}
