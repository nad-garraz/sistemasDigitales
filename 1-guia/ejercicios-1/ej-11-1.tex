\begin{enunciado}{\ejercicio}
  %Enunciado
  Represente los numeros \nBase{2}{10}, \nBase{-5}{10} y \nBase{0}{10} en notacion complemento a dos de 4 bits de
  longitud. Luego:

  \begin{enumerate}[label=(\alph*)]
    \item invierta los bits de cada representacion obtenida e indique a que numero representa en
          el mismo sistema;
    \item a partir de lo realizado en el punto anterior, proponga un metodo para obtener la representacion en
          complemento a 2 del inverso aditivo de un numero dada la representacion de ese numero en el mismo
          sistema.
  \end{enumerate}
\end{enunciado}

\begin{enumerate}[label=(\alph*)]
  \item \begin{itemize}
          \item \nBase{2}{10}
                \begin{itemize}
                  \item Complemento a dos: \nBase{0010}{2} \red{Se necesitan 3 digitos, porque con 2, solamente se podrían representar 4 numeros en total, siendo el 1 el maximo numero posible a represntar}
                  \item Inverso de la representacion obtenida: \nBase{1101}{2} =  \nBase{-3}{10}
                \end{itemize}
          \item \nBase{-5}{10}
                \begin{itemize}
                  \item Complemento a dos: \nBase{1011}{2}
                  \item Inverso de la representacion obtenida: \nBase{0100}{2} =  \nBase{4}{10}
                \end{itemize}
          \item \nBase{0}{10}
                \begin{itemize}
                  \item Complemento a dos: \nBase{0000}{2}
                  \item Inverso de la representacion obtenida: \nBase{1111}{2} =  \nBase{-1}{10}
                \end{itemize}
        \end{itemize}
  \item Metodo para obtener el inversor aditivo:
        \begin{itemize}
          \item Buscar el inverso del numero
          \item Sumar uno al inverso del numero
        \end{itemize}
\end{enumerate}

\begin{aportes}
  %\aporte{url}{nombre icono}
  \item \aporte{https://github.com/misProyectosPropios}{Iñaki Frutos \faIcon{github}}
\end{aportes}
